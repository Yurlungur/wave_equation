\documentclass[]{article}
% Time-stamp: <2013-11-20 15:21:13 (jonah)>

% packages
\usepackage{fullpage}
\usepackage{amsmath}
\usepackage{amssymb}
\usepackage{latexsym}
\usepackage{graphicx}
\usepackage{mathrsfs}
\usepackage{verbatim}
\usepackage{braket}
\usepackage{listings}
\usepackage{hyperref}

% Preamble
\hypersetup{pdftitle={background}}
\author{Jonah Miller\\
  \href{mailto:jonah.maxwell.miller@gmail.com}{jonah.maxwell.miller@gmail.com}}
\title{Wave Equation Background}

% Macros
\newcommand{\R}{\mathbb{R}}
\newcommand{\eval}{\biggr\rvert} %evaluated at

\begin{document}
\maketitle

\section{Introduction}

The one dimensional scalar wave equation takes the form
\begin{equation}
  \label{eq:1d:wave}
  \frac{\partial^2 u}{\partial t^2} = c^2 \frac{\partial^2 u}{\partial x^2},
\end{equation}
where $u$ is a scalar field defined on $x\in\Omega$ (where $\Omega$ is
some one-dimensional domain) and evolving in time for all $t\in\tau$
(where $\tau$ is an interval in $\R$). $c\in\R$ is a constant. In the
initial-boundary formulation (the problem I solve here), we specify
$u(0,x)$ and $u(t,\partial\Omega)$, where $\partial\Omega$ is the
boundary of the domain.

This is the problem I will solve. For simplicity, we will assume that
\begin{eqnarray}
  \label{eq:domains}
  \Omega &=& [0,L],\text{ where }L>0\in\R\\
  \text{and }\tau &=& [0,T],\text{ where }T>0\in\R.
\end{eqnarray}

Then the initial and boundary conditions are
\begin{eqnarray}
  \label{eq:initial:boundary:data}
  u(0,x) &=& f(x)\text{ where }f(x) \in L_2(\Omega),\\
  u(t,0) &=& \alpha,\text{ where }\alpha\in\R\\
  \text{and }u(t,L)&=&\beta\text{ where }\alpha\in\R.
\end{eqnarray}

\section{Coordinates}

We will be feeding the wave equation in to a fourth-order Runge-Kutta
integrator. Therefore, we would like to write the wave equation as a
coupled system of first-order equations, not second order. Therefore, we let
\begin{equation}
  \label{eq:r:s:def}
  s = \frac{\partial u}{\partial t}\text{ and }r = \frac{\partial u}{\partial x}.
\end{equation}
This transforms the wave equation into a first-order equation:
\begin{equation}
  \label{eq:wave:transformed:1}
  \frac{\partial s}{\partial t} = c^2 \frac{\partial r}{\partial x}.
\end{equation}
Additionally, if we take a derivative of $r$ with respect to $t$, we
get a second relationship:
\begin{eqnarray}
  \label{eq:wave:transformed:2}
  \frac{\partial r}{\partial t} &=& \frac{\partial}{\partial t}\frac{\partial }{\partial x}u = \frac{\partial }{\partial x}\frac{\partial }{\partial t}u\nonumber\\
  \Rightarrow \frac{\partial r}{\partial t} &=& \frac{\partial s}{\partial x}.
\end{eqnarray}
These two equations define the system we will integrate using
Runge-Kutta and finite difference methods. This changes the boundary
conditions somewhat. Now,
\begin{eqnarray}
  \label{eq:boundary:conditions:r:s}
  r(0,x) &=& f'(x)\\
  s(t,0) &=& s(t,L) = 0.
\end{eqnarray}
And now $\alpha$ and $\beta$ are constants of integration when we
retrieve $u$ by integrating $s$.

\section{Finite Differences}

We place the functions $u$ and $r$ on a discrete, one-dimensional grid
with fixed spacing. Let the grid points be indexed by $i$ ranging from
0 to $n$. And let the grid spacing be $h$.

This means the notion of a continuous derivative
doesn't exist. To fix this, we use a method known as finite
differences. We approximate derivatives at points on the interior by a
``centered difference'' formula and derivatives at points on the
boundary by a ``one-sided difference'' formula. Let $v$ be a scalar
field on the grid. Let each element be $v_i$, where $i$ is the index
of the grid points. Then,
\begin{equation}
  \label{eq:derivative:operator}
  D v_i = \begin{cases}
    \frac{v_{i+1} - v{i-1}}{2h}&\text{if }0 < i < n\\
    \frac{v_i-v_{i-1}}{h}&\text{if }i = n\\
    \frac{v_{i+1}-v_i}{h}&\text{if }i = 0
  \end{cases}.
\end{equation}
These formulae can easily be derived by using taylor series. See, for
example \cite{wikifinitedifferences} or \cite{Heath}.

There is a problem, however. In the continuous case, we proved the existence of solutions using a ``maximum principle'' \cite{wikimaxprinciple,BC}. And the proof of this maximum principle requires ``integration by parts'' to hold \cite{BC}:
\begin{equation}
  \label{eq:integration:by:parts}
  \braket{a,Db} + \braket{Da,b} = (ab)\eval_0^L,
\end{equation}
for all functions $a$ and $b$ in $L_2(\Omega)$, where $\braket{a,b}$
is the standard inner product on the space:
\begin{equation}
  \label{eq:inner:product}
  \braket{a,b} = \int_0^L ab dx.
\end{equation}
To show stability (which we won't), we must cook up a discrete inner
product so that a discrete version of integration by parts, called ``summation by parts,'' holds.

The key turns out to be to weigh the boundary points by half in the
inner product. The resulting inner product looks something like this:
\begin{equation}
  \label{eq:discrete:inner:product}
  \braket{a,b} = \frac{h}{2} (a_0 b_0 + a_n b_n) + h\sum_{i=1}^{n-1}a_i b_i,
\end{equation}
where $a_i$ and $b_i$ are the components of the vectors $a$ and $b$,
which exist on the discrete grid. Let's show that summation by parts does indeed hold with this inner product:

\begin{eqnarray}
  \label{eq:summation:by:parts:proof:1}
  \braket{a,Db} + \braket{Da,b} &=& \frac{1}{2}(a_0 Db_0 + a_n Db_n + b_0 Da_0 + b_n D a_n) + \sum_{i=1}^{n-1}(a_i Db_i + b_i Da_i)\nonumber\\
  &=& \frac{1}{2}\left[a_0(b_1-b_0) + b_0(a_1-a_0) + a_n(b_n-b_{n-1})+b_n(a_n-a_{n-1})\right]\nonumber\\
  &&\qquad + \frac{1}{2}\sum_{i=1}^n \left[b_i (a_{i+1}-a_{i-1}) + a_i(b_{i+1}-b_{i-1})\right]\nonumber\\
  &=& \frac{1}{2}(2 a_n b_n - 2 a_0 b_0) + \frac{1}{2}(a_0 b_1 + a_1b_0 - a_n b_{n-1} - b_n a_{n-1})\nonumber\\
  &&\qquad + \frac{1}{2} \sum_{i=1}^{n-1}\left(b_i a_{i+1} - b_i a_{i-1} + a_i b_{i+1} - a_i b_{i-1}\right)
\end{eqnarray}
Notice that in equation \eqref{eq:summation:by:parts:proof:1}, the
$\frac{1}{2}(2 a_n b_n - 2 a_0 b_0)$ term is equal to $(ab)\eval_0^L.$
Furthermore, if we expand the sum from $i=1$ to $i=n-1$, it will
collapse, leaving only the initial and final terms. And these terms
cancel perfectly with the $\frac{1}{2}(a_0 b_1 + a_1b_0 - a_n b_{n-1}
- b_n a_{n-1})$ terms. So, in the end, we're left with
\begin{equation}
  \label{eq:summation:by:parts:proof:2}
  \braket{a,Db}+\braket{Da,b} = (ab)\eval_0^L.
\end{equation}
This means that our inner product is stable. For more information, see \cite{Diener,Olsson1,Olsson1Sup,Olsson2}.

\section{Numerical Methods}

When we put $r$ and $s$ on a grid, we generate two $n$-dimensional
vectors that are related by the equations discussed above. To solve
for the time evolution of these vectors, we can feed them into a
Runge-Kutte integrator. For more information on Runge Kutte methods,
see, for example, \cite{Heath}.

\begin{thebibliography}{8}

\bibitem{wikifinitedifferences} Wikipedia contributors. Finite difference. Wikipedia, The Free Encyclopedia. November 13, 2013, 20:42 UTC. Available at: \url{http://en.wikipedia.org/w/index.php?title=Finite_difference&oldid=581529514.} Accessed November 20, 2013. 

\bibitem{Heath} Heath, Michael T. \textit{Scientific Computing: An Introductory Survey.} United States: McGraw-Hill; 1997.

\bibitem{wikimaxprinciple} Wikipedia contributors. Maximum principle. Wikipedia, The Free Encyclopedia. March 23, 2013, 09:10 UTC. Available at: \url{http://en.wikipedia.org/w/index.php?title=Maximum_principle&oldid=546502732.} Accessed November 20, 2013. 

\bibitem{BC} Bleeker, David D. and Csordas, George. \textit{Basic Partial Differential Equations.} Boston, MA: International Press of Boston; 1996.

\bibitem{Diener} Diener, P, Dorband, E, Schnetter, and E, Tiglio, M. ``New, efficient, and accurate high order derivatives and dissipation operators satisfying summation by parts, and applications in three-dimensional multi-block evolutions.'' \href{arXiv:gr-qc/0512001}{http://arxiv.org/abs/gr-qc/0512001}. (2005). Online resource.

\bibitem{Olsson1} Olsson, Pelle. ``Summation by Parts, Projections, and Stability. I.'' \textit{Mathematics of Computation}. 64.11 (1995): 1035--1065. Online resource.

\bibitem{Olsson1Sup} Olsson, Pelle. ``Supplement to Summation by Parts, Projections, and Stability. I.'' \textit{Mathematics of Computation}. 64.211 (1995): S23--S26. Online resource.

\bibitem{Olsson2} Olsson, Pelle. ``Summation by Parts, Projections, and Stability II.'' \textit{Mathematics of Computation.} 64.212 (1995): 1473--1493. Online resource.

\end{thebibliography}

\end{document}
